\subsection{Synopsys}

\begin{verbatim}
Inst ( model
     , name
     , map = myMap
     )
\end{verbatim}

\subsection{Description}

Instantiation of an instance. The type of the instance is given by the \verb-model- parameter. The connexions are made thanks to the \verb-map- parameters.

\subsection{Parameters}

\begin{itemize}
    \item \verb-model- : Name of the mastercell of the instance to create (mandatory argument)
    \item \verb-name- : Name of the instance (optional)\\
When this argument is not defined, the instance has a name created by default. This argument is usefull when one wants to create a layout as well. Indeed, the placement of the instances is much easier when the conceptor has chosen himself the name f the instances.
    \item \verb-map- : Dictionnary for connexions in order to make the netlist\\
\end{itemize}

\subsection{Attributes}

\begin{itemize}
    \item \verb-_name- : Name of the instance (the name given as parameter if there's one, a name created otherwise)
    \item \verb-_model- : Name of the model given as argument
    \item \verb-_real_model- : Name of the model created thanks to \verb-_model- and all the parameters
    \item \verb-_map- : Dictionnary \verb-map- given at the instanciation
    \item \verb-_param- : Dictionnary \verb-param- given at the instanciation
    \item \verb-_st_cell- : The stratus cell which the instance is instanciated in
    \item \verb-_st_masterCell- : The stratus master cell of the instance\\
\end{itemize}
\indent For placement :
\begin{itemize}
    \item \verb-_plac- : tells if the instance is placed or not (UNPLACED by default)
    \item \verb-_x-, \verb-_y- : the coordinates of the instance (only for placed instances)
    \item \verb-_sym- : the symetry of the instance (only for placed instances)\\
\end{itemize}
\indent And, in connection with Hurricane :
\begin{itemize}
    \item \verb-_hur_instance- : The hurricane instance (None by default)
    \item \verb-_hur_masterCell- : The Hurricane master cell of the instance (None by default)
\end{itemize}

\subsection{Methods}

\begin{itemize}
    \item Delete : Deletion of the Hurricane instance
\end{itemize}
